\documentclass[12pt]{ociamthesis}\usepackage[]{graphicx}\usepackage[]{color}
%% maxwidth is the original width if it is less than linewidth
%% otherwise use linewidth (to make sure the graphics do not exceed the margin)
\makeatletter
\def\maxwidth{ %
  \ifdim\Gin@nat@width>\linewidth
    \linewidth
  \else
    \Gin@nat@width
  \fi
}
\makeatother

\definecolor{fgcolor}{rgb}{0.345, 0.345, 0.345}
\newcommand{\hlnum}[1]{\textcolor[rgb]{0.686,0.059,0.569}{#1}}%
\newcommand{\hlstr}[1]{\textcolor[rgb]{0.192,0.494,0.8}{#1}}%
\newcommand{\hlcom}[1]{\textcolor[rgb]{0.678,0.584,0.686}{\textit{#1}}}%
\newcommand{\hlopt}[1]{\textcolor[rgb]{0,0,0}{#1}}%
\newcommand{\hlstd}[1]{\textcolor[rgb]{0.345,0.345,0.345}{#1}}%
\newcommand{\hlkwa}[1]{\textcolor[rgb]{0.161,0.373,0.58}{\textbf{#1}}}%
\newcommand{\hlkwb}[1]{\textcolor[rgb]{0.69,0.353,0.396}{#1}}%
\newcommand{\hlkwc}[1]{\textcolor[rgb]{0.333,0.667,0.333}{#1}}%
\newcommand{\hlkwd}[1]{\textcolor[rgb]{0.737,0.353,0.396}{\textbf{#1}}}%
\let\hlipl\hlkwb

\usepackage{framed}
\makeatletter
\newenvironment{kframe}{%
 \def\at@end@of@kframe{}%
 \ifinner\ifhmode%
  \def\at@end@of@kframe{\end{minipage}}%
  \begin{minipage}{\columnwidth}%
 \fi\fi%
 \def\FrameCommand##1{\hskip\@totalleftmargin \hskip-\fboxsep
 \colorbox{shadecolor}{##1}\hskip-\fboxsep
     % There is no \\@totalrightmargin, so:
     \hskip-\linewidth \hskip-\@totalleftmargin \hskip\columnwidth}%
 \MakeFramed {\advance\hsize-\width
   \@totalleftmargin\z@ \linewidth\hsize
   \@setminipage}}%
 {\par\unskip\endMakeFramed%
 \at@end@of@kframe}
\makeatother

\definecolor{shadecolor}{rgb}{.97, .97, .97}
\definecolor{messagecolor}{rgb}{0, 0, 0}
\definecolor{warningcolor}{rgb}{1, 0, 1}
\definecolor{errorcolor}{rgb}{1, 0, 0}
\newenvironment{knitrout}{}{} % an empty environment to be redefined in TeX

\usepackage{alltt}  % default square logo 
%\documentclass[12pt,beltcrest]{ociamthesis} % use old belt crest logo
%\documentclass[12pt,shieldcrest]{ociamthesis} % use older shield crest logo

%load any additional packages
\usepackage{amssymb}
\usepackage[english]{babel}
\usepackage{graphicx}
\usepackage{url}
\usepackage{lipsum}
\usepackage{array}
\usepackage{float}
\usepackage[%
  backend=bibtex      % biber or bibtex
 ,style=numeric-comp    % Alphabeticalsch
 %,style=numeric-comp  % numerical-compressed
 ,sorting=none        % no sorting
 ,sortcites=true      % some other example options ...
 ,block=none
 ,indexing=false
 ,citereset=none
 ,isbn=true
 ,url=true
 ,doi=true            % prints doi
 ,natbib=true         % if you need natbib functions
]{biblatex}

\addbibresource{refs.bib} %Imports bibliography file

\newcommand{\cdiff}{C. \textit{diff}}

%input macros (i.e. write your own macros file called mymacros.tex 
%and uncomment the next line)
%\include{mymacros}

%note \\[1ex] is a line break in the title
\title{A Longitudinal Study of the Effect of Renal Failure on Readmission Rates of Patients with \textit{Clostridium Difficile}}

\author{Brian Detweiler}
\college{College of Arts and Sciences}  %your college

%\renewcommand{\submittedtext}{change the default text here if needed}
\degree{Master of Science}     %the degree
\degreedate{May 13, 2017}         %the degree date

%end the preamble and start the document
\IfFileExists{upquote.sty}{\usepackage{upquote}}{}
\begin{document}

%this baselineskip gives sufficient line spacing for an examiner to easily
%markup the thesis with comments
\baselineskip=18pt plus1pt

%set the number of sectioning levels that get number and appear in the contents
\setcounter{secnumdepth}{3}
\setcounter{tocdepth}{3}


\maketitle                  % create a title page from the preamble info

% include a dedication.Rnw file
%<<dedication, child='dedication.Rnw'>>=
%@
 
% include an acknowledgements.Rnw file%
%<<acknowledgements, child='acknowledgements.Rnw'>>=
%@

% include the abstract

\begin{abstract}
Clostridium Difficile Infection (C. \textit{diff}, or simply CDI) 
is a highly contagious endospore forming bacterium that is transferred
through physical contact with an infected surface. Symptoms range from diarrhea to
life-threatening colitis and is most commonly acquired in a hospital setting where
antimicrobials have been administered. Increased mortality in
CDI patients with renal failure comorbidities was shown in the literature as early as 1998 \cite{Cunney1998}.
In this study, we use the \textit{Nationwide Readmissions Database} to assess the risk of 
30, 60, and 90 day readmissions in patients with comorbid 
CDI and renal failure conditions. We also discuss general CDI trends from 2001-2014, using the 
\textit{Nationwide Inpatient Sample}. 
\end{abstract}

\begin{romanpages}          % start roman page numbering
\tableofcontents            % generate and include a table of contents
\listoffigures              % generate and include a list of figures
\end{romanpages}            % end roman page numbering





%now include the files of latex for each of the chapters etc

\chapter{Introduction}

Clostridium Difficile Infection (often referred to as C. \textit{diff}, or CDI), 
has been an increasing concern in hospitals and nursing homes, where it is most frequently acquired.
\cite{Lamont2017}
\cite{Lessa2015}

\section{Literature review}

\lipsum

\chapter{Methods}


\section{Data source}

The Agency for Healthcare Research and Quality (AHRQ), under the Department of Health and Human Services (DHHS), sponsors the
Healthcare Cost and Utilization Project (HCUP), a collection of databases including the Nationwide Inpatient Sample (NIS) and
the Nationwide Readmissions Database (NRD). \cite{HCUPOverview}

\subsection{Nationwide Inpatient Sample}
The NIS is an annual survey of inpatient discharges dating back to 1988. 

\subsection{Nationwide Readmissions Database}

\section{Study sample}



\section{Statistical analysis}

\chapter{Results}




\section{Trends}

\cdiff has been on the rise since the first reported major outbreak of ribotype 027 in 2004 \cite{Pepin2004}. 

has  the elderly, and that continues to be the case today. Caroll and Bartlett \cite{Carroll2011}
Figure \ref{by_age} shows the distribution of \cdiff patients by age for all data from 2001-2014. 
\cite{Masgala2014}


In Figure \ref{age_dist_over_time}, we plot the



\begin{knitrout}
\definecolor{shadecolor}{rgb}{0.969, 0.969, 0.969}\color{fgcolor}\begin{figure}

{\centering \includegraphics[width=\maxwidth]{figure/by_age-1} 

}

\caption[The blue lines on either side represent the interquartile range, while the red line in the middle ]{The blue lines on either side represent the interquartile range, while the red line in the middle }\label{fig:by_age}
\end{figure}


\end{knitrout}


Increased age as a risk factor may be confounded by increased risk of other acquired comorbidities such as renal failure. 
\cite{Krapohl2013} \cite{Masgala2014}



\begin{knitrout}
\definecolor{shadecolor}{rgb}{0.969, 0.969, 0.969}\color{fgcolor}\begin{figure}

{\centering \includegraphics[width=\maxwidth]{figure/age_dist_over_time-1} 

}

\caption[Age distribution by year from 2001 to 2014]{Age distribution by year from 2001 to 2014. The distribution is becoming less left-skewed and more platykurtic.}\label{fig:age_dist_over_time}
\end{figure}


\end{knitrout}


\begin{knitrout}
\definecolor{shadecolor}{rgb}{0.969, 0.969, 0.969}\color{fgcolor}\begin{kframe}


{\ttfamily\noindent\itshape\color{messagecolor}{\#\# Don't know how to automatically pick scale for object of type ts. Defaulting to continuous.}}\end{kframe}\begin{figure}

{\centering \includegraphics[width=\maxwidth]{figure/age_groups_over_time-1} 

}

\caption[Age distribution by year from 2001 to 2014]{Age distribution by year from 2001 to 2014. }\label{fig:age_groups_over_time}
\end{figure}


\end{knitrout}

\chapter{Discussion}

\lipsum

\chapter{Conclusion and future work}

\lipsum

\chapter{Conclusion}

%now enable appendix numbering format and include any appendices
%\appendix
%<<appendix1, child='appendix1.Rnw'>>=
%@
%<<appendix2, child='appendix2.Rnw'>>=
%@

%next line adds the Bibliography to the contents page
%\addcontentsline{toc}{chapter}{Bibliography}
%uncomment next line to change bibliography name to references
%\renewcommand{\bibname}{References}
%\bibliography{refs}        %use a bibtex bibliography file refs.bib
% \bibliographystyle{plain}  %use the plain bibliography style
% \bibliographystyle{apalike}
\renewcommand{\bibname}{References}
\printbibliography
%Sets the bibliography style to UNSRT and imports the 
%bibliography file "samples.bib".
\end{document}

\documentclass[12pt]{ociamthesis}\usepackage[]{graphicx}\usepackage[]{color}
%% maxwidth is the original width if it is less than linewidth
%% otherwise use linewidth (to make sure the graphics do not exceed the margin)
\makeatletter
\def\maxwidth{ %
  \ifdim\Gin@nat@width>\linewidth
    \linewidth
  \else
    \Gin@nat@width
  \fi
}
\makeatother

\definecolor{fgcolor}{rgb}{0.345, 0.345, 0.345}
\newcommand{\hlnum}[1]{\textcolor[rgb]{0.686,0.059,0.569}{#1}}%
\newcommand{\hlstr}[1]{\textcolor[rgb]{0.192,0.494,0.8}{#1}}%
\newcommand{\hlcom}[1]{\textcolor[rgb]{0.678,0.584,0.686}{\textit{#1}}}%
\newcommand{\hlopt}[1]{\textcolor[rgb]{0,0,0}{#1}}%
\newcommand{\hlstd}[1]{\textcolor[rgb]{0.345,0.345,0.345}{#1}}%
\newcommand{\hlkwa}[1]{\textcolor[rgb]{0.161,0.373,0.58}{\textbf{#1}}}%
\newcommand{\hlkwb}[1]{\textcolor[rgb]{0.69,0.353,0.396}{#1}}%
\newcommand{\hlkwc}[1]{\textcolor[rgb]{0.333,0.667,0.333}{#1}}%
\newcommand{\hlkwd}[1]{\textcolor[rgb]{0.737,0.353,0.396}{\textbf{#1}}}%
\let\hlipl\hlkwb

\usepackage{framed}
\makeatletter
\newenvironment{kframe}{%
 \def\at@end@of@kframe{}%
 \ifinner\ifhmode%
  \def\at@end@of@kframe{\end{minipage}}%
  \begin{minipage}{\columnwidth}%
 \fi\fi%
 \def\FrameCommand##1{\hskip\@totalleftmargin \hskip-\fboxsep
 \colorbox{shadecolor}{##1}\hskip-\fboxsep
     % There is no \\@totalrightmargin, so:
     \hskip-\linewidth \hskip-\@totalleftmargin \hskip\columnwidth}%
 \MakeFramed {\advance\hsize-\width
   \@totalleftmargin\z@ \linewidth\hsize
   \@setminipage}}%
 {\par\unskip\endMakeFramed%
 \at@end@of@kframe}
\makeatother

\definecolor{shadecolor}{rgb}{.97, .97, .97}
\definecolor{messagecolor}{rgb}{0, 0, 0}
\definecolor{warningcolor}{rgb}{1, 0, 1}
\definecolor{errorcolor}{rgb}{1, 0, 0}
\newenvironment{knitrout}{}{} % an empty environment to be redefined in TeX

\usepackage{alltt}  % default square logo 
%\documentclass[12pt,beltcrest]{ociamthesis} % use old belt crest logo
%\documentclass[12pt,shieldcrest]{ociamthesis} % use older shield crest logo

%load any additional packages
\usepackage{amssymb}
\usepackage[english]{babel}
\usepackage{graphicx}
\usepackage{url}
\usepackage{lipsum}
\usepackage{array}
\usepackage{float}

%input macros (i.e. write your own macros file called mymacros.tex 
%and uncomment the next line)
%\include{mymacros}

\title{Clostridium Difficile\\[1ex]     %your thesis title,
       (Working Title)}   %note \\[1ex] is a line break in the title

\author{Brian Detweiler}
\college{College of Arts and Sciences}  %your college

%\renewcommand{\submittedtext}{change the default text here if needed}
\degree{Master of Science}     %the degree
\degreedate{May 13, 2017}         %the degree date

%end the preamble and start the document
\IfFileExists{upquote.sty}{\usepackage{upquote}}{}
\begin{document}

%this baselineskip gives sufficient line spacing for an examiner to easily
%markup the thesis with comments
\baselineskip=18pt plus1pt

%set the number of sectioning levels that get number and appear in the contents
\setcounter{secnumdepth}{3}
\setcounter{tocdepth}{3}


\maketitle                  % create a title page from the preamble info

% include a dedication.Rnw file
%<<dedication, child='dedication.Rnw'>>=
%@
 
% include an acknowledgements.Rnw file%
%<<acknowledgements, child='acknowledgements.Rnw'>>=
%@

% include the abstract

\begin{abstract}
Clostridium Difficile Infection (CDI) is a highly contagious endospore forming bacterium that is transferred
through physical contact with an infected surface. Symptoms range from diarrhea to
life-threatening colitis. Elderly patients in hospitals and nursing homes receiving antimicrobial
treatment are at the highest risk of infection, and roughly one in nine patients over 65 will
die within 30 days of diagnosis. In this paper, we seek to model the risk of contracting CDI in both
hospital and nursing home settings given patient-to-patient or patient-to-healthcare provider 
contact. We then provide recommendations for minimizing the risk and controlling the spread of CDI.
\end{abstract}

\begin{romanpages}          % start roman page numbering
\tableofcontents            % generate and include a table of contents
\listoffigures              % generate and include a list of figures
\end{romanpages}            % end roman page numbering

%now include the files of latex for each of the chapters etc

\chapter{Introduction}

\lipsum


\chapter{CDI and stuff}

\lipsum

\section{More stuff}

\lipsum

\chapter{Some more stuff}

\lipsum

\chapter{Even more stuff}

\lipsum

\chapter{The most stuff}

\lipsum

\chapter{Conclusion}

%now enable appendix numbering format and include any appendices
%\appendix
%<<appendix1, child='appendix1.Rnw'>>=
%@
%<<appendix2, child='appendix2.Rnw'>>=
%@

%next line adds the Bibliography to the contents page
\addcontentsline{toc}{chapter}{Bibliography}
%uncomment next line to change bibliography name to references
%\renewcommand{\bibname}{References}
\bibliography{thesis}        %use a bibtex bibliography file refs.bib
% \bibliographystyle{plain}  %use the plain bibliography style
\bibliographystyle{unsrt}

%Sets the bibliography style to UNSRT and imports the 
%bibliography file "samples.bib".
\end{document}

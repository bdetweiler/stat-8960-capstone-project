\chapter{Introduction}

\section{Overview of \cdiff}

Clostridium \textit{difficile} Infection - also referred to by its shortened name, \cdiff, or simply CDI -
has been an increasing concern in the last two decades among healthcare providers. 
The organism itself is a resilliant endospore-forming bacterium, resistant to heat, acid, and antibiotics,
and can survive on surfaces for up to 5 months, if proper sanitation is not carried out. \cite{Gerding2008}

In past years, the most common \cdiff cases occurred in elderly patients, 65 years or older,
who were on admitted as an inpatients in a hospital or nursing home setting, and given antimicrobial therapy.
In fact, it is the most frequently 
Antimicrobials deplete the healthy flora lining the intestines which 

Adding to the complexity of the situation, CDI carriers can remain asymptomatic, making them stealth transporters
and allowing the disease to propegate undetected until it is too late. 

In 2013, the CDC estimated that around 250,000 Americans contracted \cdiff in a single year, causing 14,000 deaths.
That estimate was later updated to half a million in 2015, causing 15,000 deaths. \cite{CDC2018}
\cite{CDC2015} 
Another study puts that number even higher, at 29,000 deaths in 2011. 

\cdiff is also costly. The CDC estimates that in 2008, it cost acute healthcare facilities alone more than \$4.8 billion.
The mean cost of an incident of CDI was found to be \$11,498 (inflation adjusted to 2008 dollars)
and as high as \$15,397 when CDI was hospital acquired. \cite{Dubberke2012}


\section{Overview of Renal Failure}

While \cdiff is a singular diagnosis category, renal failure falls into one of two umbrella categories, acute kidney injury (AKI), 
and chronic kidney disease (CKD). AKI is further broken into subcategories. Acute tubular necrosis is the most common form of AKI.
Other subcategories include renal cortical necrosis, renal medullary necrosis, lesions, and a category for unspecified AKI. 

Chronic kidney disease is broken into categories based on stages that are calculated using the estimated Glomerular Filtration Rate
equation.

\begin{equation*}
\begin{split}
  GFR \footnote{GFR is in mL/min/1.73 m2} &= 175 * Scr - 1.154 * Age - 0.203 * (0.742 * I(female)) * (1.212 I(\text{African American})) \\
\end{split}
\end{equation*}



\section{Readmissions}

A 2014 study done by the Agency for Healthcare Research and Quality (AHRQ) under the Healthcare Cost and Utilization
Project (HCUP) found hospital readmissions accounted for about \$41.3 billion in hospital costs.  \cite{Hines2014}

Under the Readmission Reduction Program, a provision of the Affordable Care Act, 
Hospitals face penalties on Medicare payments if they exceed certain 30-day readmission standards. While the 
American Hospital Association strongly opposes the measure, citing a lack of control over the chain of events
that can lead to readmission \cite{Rice2015} \cite{AHA2018}, the Affordable Care Act is still the rule of law, and hospitals must seek
to reduce readmissions in order to avoid penalties.

For this reason, readmission statistics are an important key metric for hospitals interested in optimizing 
their operations. Using large surveys, researchers are able to determine trends and end results, but not necessarily 
causes, of readmissions. Still, high level trends can point healthcare providers in a direction where they can
more efficiently focus their attention. 





\cite{Lessa2015}

\section{Overview of the data}

\subsection{The Nationwide Inpatient Sample (NIS)}

\subsection{The Nationwide Readmissions Database (NRD)}



\cite{Lamont2017}
\cite{Lessa2015}

\section{Literature review}

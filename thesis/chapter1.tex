\chapter{Introduction}

\section{Overview of \cdifficile}

\cdifficile, or \cdiff Infection (CDI) -
has been an increasing concern in the last two decades among healthcare providers. 
The organism itself is a resilient endospore-forming bacterium, resistant to heat, acid, and antibiotics,
and can survive on surfaces for up to 5 months, if proper sanitation is not carried out \cite{Gerding2008}.

In past years, the most common CDI cases occurred in elderly patients, 65 years or older,
who were admitted as inpatients in a hospital or nursing home setting, and given antimicrobial therapy.
In fact, it has become the most frequent nosocomial (hospital-acquired) disease, surpassing
methicillin-resistant Staphylococcus aureus (MRSA) \cite{Gupta2014}.

Antimicrobials deplete the healthy gut flora of the intestines which protect against harmful organisms like \cdiff \cite{Lamont2017}.

Adding to the complexity of the situation, \cdiff carriers can remain asymptomatic, making them stealth transporters
and allowing the disease to propagate undetected until it is too late. 

In 2013, the CDC estimated that around 250,000 Americans contracted CDI in a single year, causing 14,000 deaths.
That estimate was later updated to half a million in 2015, causing 15,000 deaths \cite{CDC2015, CDC2018}.
Another study puts that number even higher, at 29,000 deaths in 2011. 

CDI is also costly. The CDC estimates that in 2008, it cost acute healthcare facilities alone more than \$4.8 billion.
The mean cost of an incident of CDI was found to be \$11,498 (inflation adjusted to 2008 dollars)
and as high as \$15,397 when CDI was hospital acquired \cite{Dubberke2012}.


\section{Overview of Renal Failure}

While CDI is a singular diagnosis category, renal failure falls into one of two umbrella categories, acute kidney injury (AKI), 
and chronic kidney disease (CKD). AKI is further broken into subcategories. Acute tubular necrosis is the most common form of AKI.
Other subcategories include renal cortical necrosis, renal medullary necrosis, lesions, and a category for unspecified AKI. 

When evaluating the effect of renal failure on outcome variables, we first need to understand how renal failure is classified.

Chronic kidney disease is broken into categories based on stages that are calculated using one of the estimated Glomerular Filtration Rate
(GFR) equations. The equations model kidney health as a function of age, sex, race, and blood creatinine, a waste product that is produced from
normal muscle use. One of two formulas may be used, the Modification of Diet in Renal Disease (MDRD) formula, or the newer
Chronic Kidney Disease Epidemiology Collaboration (CKD-EPI) formula \cite{eGFR2018}. The models are included below to show the roles that
age, sex, and race play in the diagnosis of CKD.

\subsection{The MDRD Equation vs. the CKD-EPI Equation}

Pateints are classified as having a particular CKD stage by testing their Glomerular Filtration Rate (GFR) using one of two equations.
Both equations measure the creatinine in a patient's blood with a serum creatinine test. Creatinine is a waste product produced by normal
muscle wear and tear. 

The MDRD equation encodes sex and race (African American or not) and does not rely on height or weight due to using $1.73m^2$ surface area,
the generally accepted mean human adult body surface area. 

\begin{equation} \label{mdrd}
\begin{split}
  GFR  &= 175 \times S_{cr} - 1.154 \times \text{Age}^{-0.203} \times 0.742 \cdot I(\text{F}) \times 1.212 \cdot I(\text{AA}) \\
\end{split}
\end{equation}

where:
\begin{itemize}
  \item F is female sex
  \item AA is African American race
  \item I is an indicator function that returns 1 if true, the reciprocal of the preceding term if false (thereby making the preceding term 1)
  \item $S_{cr}$ is serum creatinine in mg/dL
\end{itemize} \cite{Levey1999}

The CKD Epidemiology (CKD-EPI) was a single equation selected out of a large number of candidate equations, that uses transformations of continuous
variables and additional variables and interactions. Serum creatinine is modeled as a 2-slope spline with sex-specific knots at 0.7 mg/dL for women 
and 0.9 mg/dL for men. It was shown to outperform the MDRD, with lower bias and increased precision.  \cite{Levey2009, eGFR2018}

\begin{equation} \label{ckdepi}
\begin{split}
  GFR &= 141 \times min\bigg(\frac{S_{cr}}{\kappa}, 1\bigg)^{\alpha} \times max\bigg(\frac{S_{cr}}{\kappa}, 1\bigg)^{-1.209} \\
      &\times 0.993^{\text{Age}} \times 1.018 \cdot \text{I}(\text{F}) \times 1.159 \cdot \text{I}(\text{AA}) \\
\end{split}
\end{equation}

where:
\begin{itemize}
  \item $\kappa$ is 0.7 for females and 0.9 for males
  \item $\alpha$ is -0.329 for females and -0.411 for males
\end{itemize} \cite{Levey2009}

\subsection{Classifying Chronic Kidney Disease}

Once the GFR is calculated using either the MDRD or the CKD-EPI formulas, 
patients can be placed into one of five categories. Table \ref{tab:gfr} shows the 
GFR rating along with the stage of CKD and the level of kidney function. \textbf{585} is used in the ICD-9-CM coding system
to indicate CKD. If the level is known, a more specific coding is used. \textbf{585.1-585.5} indicate CKD Stage 1 through 5.
\textbf{585.6} is used for end stage renal disease (dialysis or transplant), and \textbf{585.9} is used if the level is unspecified.

\begin{table}[]
\centering
\label{my-label}
\begin{tabular}{lllll}
CKD Stage & Description             & GFR        & Kidney Function & ICD-9-CM \\
\hline
1         & Normal function         & 90+        & 90-100\%        & 585.1    \\
2         & Mild loss               & 60-89      & 60-89\%         & 585.2    \\
3         & Mild to severe          & 30-59      & 30-59\%         & 585.3    \\
4         & Severe                  & 15-29      & 15-29\%         & 585.4    \\
5         & Kidney failure          & 15 or less & 15\% or less    & 585.5    \\
\end{tabular}\caption{GFR classifications for stages of Chronic Kidney Disease}\label{tab:gfr}
\end{table}

\subsection{End stage renal disease}

When CKD reaches Stage 5, it is considered end stage renal disease (ESRD). Dialysis or a kidney transplant is needed to stay alive.
The most common causes of ESRD are diabetes and high blood pressure. Risk of ESRD also increases with age. 


\section{Index Admissions and Readmissions}

An inpatient admission begins on the first day a patient is admitted to a hospital under a doctor's order. The 
last day before discharge is the last inpatient day \cite{Medicare}.

Readmissions are subsequent admissions from a given \textit{index} admission within a specified time interval.
Methods and inclusion/exclusion criteria for determining an index admission and readmissions vary. 

\subsection{Readmissions and the ACA}

A 2014 study done by the Agency for Healthcare Research and Quality (AHRQ) under the Healthcare Cost and Utilization
Project (HCUP) found hospital readmissions accounted for about \$41.3 billion in hospital costs \cite{Hines2014}.

Under the Readmission Reduction Program, a provision of the Affordable Care Act, 
Hospitals face penalties on Medicare payments if they exceed certain 30-day readmission standards. While the 
American Hospital Association strongly opposes the measure, citing a lack of control over the chain of events
that can lead to readmission \cite{Rice2015, AHA2018}, the Affordable Care Act is still the rule of law, and hospitals must seek
to reduce readmissions in order to avoid penalties.

For this reason, readmission statistics are an important key metric for hospitals interested in optimizing 
their operations. Using large surveys, researchers are able to determine trends and end results, but not necessarily 
causes, of readmissions. Still, high level trends can point healthcare providers in a direction where they can
more efficiently focus their attention. Studies like this one focus on narrow cases where a better understanding can 
contribute to reduced readmissions and an overall reduction in penalties.

\subsection{Readmission measures}

The Centers for Medicare \& Medicaid Services (CMS) sets guidelines that hospitals must follow to avoid penalization on Medicare payments. 
The CMS measures "excess readmissions" as a ratio of predicted-to-expected readmissions and each hospital's relative performance, based on
a 30-day risk standardized measure. All-cause unplanned readmissions to the same or another applicable acute care hospital, 
occurring within 30 days - for any reason, regardless of principal diagnosis - from the index admission are counted in this measure.
Some planned readmissions are not counted \cite{HRRP}. 

For fiscal years 2013 to 2018, the following formula is used to calculate the Payment Readjustment Factor (PRF):

\begin{equation} \label{prf}
\begin{split}
  \text{PRF} &= 1 - min\bigg(0.03, \sum_{dx} \frac{\text{Payment}(dx) \cdot max\big((\text{ERR}(dx) - 1.0), 0\big)}{\text{All payments}}\bigg) \\
\end{split}
\end{equation}
 
Where $dx$ is one of six measure cohorts: 

\begin{itemize}
  \item acute myocardial infarction (AMI)
  \item heart failure (HF)
  \item pneumonia
  \item chronic obstructive pulmonary disease (COPD)
  \item coronary artery bypass graft (CABG) surgeries
  \item elective primary total hip and/or total knee arthroplasty (THA/TKA)
\end{itemize}

ERR is a hospital's performance measure against $dx$, and payment refers to base operating DRG payments \cite{HRRPPaymentAdjustment, Lessa2015}.


\section{Overview of the data}

The data obtained for this study comes from the Agency for Healthcare Research and Quality (AHRQ),
under the Department of Health and Human Services (DHHS). ARHQ sponsors the
Healthcare Cost and Utilization Project (HCUP), a collection of databases including the Nationwide Inpatient Sample (NIS) and
the Nationwide Readmissions Database (NRD) \cite{HCUPOverview}. For this study, we obtained years 2001-2014 of the NIS, and
2010-2014 of the NRD.

Both datasets are based on complex survey designs. The Primary Sampling Units (PSUs) are hospitals, stratified by region,
teaching status, and bedsize. Weights are calculated for each discharge which are used to "map" the sample back to an unbiased
representation of the survey population \cite{Heeringa2017}.

\subsection{The Nationwide Inpatient Sample (NIS)}

The Agency for Healthcare Research (AHRQ) has been conducting the National (later renamed to "Nationwide") Inpatient
Sample since 1988, as part of the Healthcare Cost and Utilization Project (HCUP). It estimates a weighted 35 million
hospitalizations per calendar year using around 7-8 million unweighted discharges per year. It is the largest database
of its kind in the United States \cite{NISOverview}.

\subsection{The Nationwide Readmissions Database (NRD)}

Similar to the NIS, the NRD tracks hospitalizations. In addition, it tracks patients across admissions, using an ID key,
an admission reference date, and a length of stay for each admission. This allows analysts to track anonymized readmission cases.
The NRD tracks around 14 million unweighted patients, when weighted, estimates about 36 million weighted patients across admissions 
per calendar year \cite{NRDOverview}.

\subsection{Limitations}

Working on such a rich dataset does not come without limits. Prior to obtaining the NIS or NRD, analysts must take the HCUP
Data Usage Agreement (DUA). At a high level, HCUP requires that researchers protect individual identities. Cell sizes (groups of people)
where $n \le 10$ may not be reported. Attempting to identify individual patients or health care providers through vulnerability or penetration testing, 
or any other means, is prohibited. For instance, if an analyst noticed that a certain hospital used a particular coding scheme that was unique 
to that hospital and could be used to identify aspects of a patient, the analyst may not use this information. Furthermore, probing for such
vulnerabilities is prohibited as well. Publication of any methodology that could identify individuals is prohibited.

Furthermore, HCUP data may only be used for research, not for commercial or competitive purposes. Institutions may not be contacted
to verify any of the data within the HCUP datasets either. 

For the above reasons, the data may also not be posted online, and anyone wishing to work with or even see the data must take the DUA class and
sign the agreement \cite{HCUPDUA}.

In a couple of recent publications, Khera and Krumholz expanded on these base requirements and offered a checklist \cite{Krumholz2017} 
for analysts to follow as they work with the NIS. In a followup study, they found only 10.5\% (95\% CI, 4.7\%-16.4\%) of published research
projects based on NIS data followed all of the guidelines \cite{Khera2017}.

Listed below are the guidelines from Khera and Krumholz, and notes on how we have conformed to them.


\begin{itemize}
  \item  \textbf{Section A: Research Design}
  \begin{todolist}
  \item[\done] Does the study consider that it can only detect disease conditions, procedures, and diagnostic tests in hospital settings?
  
  \textbf{Yes, we make no assumptions about events occurring outside of the hospital setting.}
  
  \item[\done] Does the study acknowledge that it includes encounters, not individual patients?
  
  \textbf{Yes, all of our assumptions are made upon the basis of inpatient discharges and readmissions, not individuals.}
  
  \item[\done] Does the study avoid diagnosis/procedure-specific volume assessments for units that are not part of the
  sampling frame of the NIS, and are therefore not representatively sampled, including
  
  \begin{itemize}
    \item geographic units, like U.S. states
    \item healthcare facilities (after 2011)
    \item individual healthcare providers?
  \end{itemize} 
          
  \textbf{Yes, we only make assessments at the national level.}
  
  \end{todolist}
  
  \item  \textbf{Section B: Data Interpretation}
  \begin{todolist}
  \item[\done] Does the study attempt to identify disease conditions or procedures of interest using administrative 
  codes or their combinations that have been previously validated?
 
  \textbf{Yes, when checking for renal failure, the comorbidity indicators (\texttt{renlfail, cm\_renlfail}) were used 
  before assessing ICD-9-CM codes.}
  
  \item[\done] Does the study limit its assessments to only in-hospital outcomes, rather than those occurring after discharge?
  
  \textbf{Yes, the only outcome assessments were readmission status and mortality (\texttt{died}), both of which are in-hospital events.}
  
  \item[\done] Does the study distinguish complications from comorbidities or clearly note where it cannot?
  
  \textbf{Yes, renal failure comorbidities were distinguished using the the comorbidity indicators (\texttt{renlfail, cm\_renlfail}). 
  CDI while most often a complication, cannot usually be distinguished between complication and comorbidity however.}
  
  
  \end{todolist}
  
  \item  \textbf{Section C: Data Analysis}
  \begin{todolist}
  \item[\done] Does the study clearly account for the survey design of the NIS and its components -clustering, stratification, and weighting?
 
  \textbf{Yes, the R \texttt{survey} package was used to account for survey design.}
  
  \item[\done] Does the study adequately address changes in data structure over time (for trend analyses)?
  
  \textbf{Yes, since we are only doing national-level assessments, and we are not using ICD-10-CM codes in the 2015 datasets, 
  we don't need to worry about the changes in the survey design.}
  
  \end{todolist}
  
\end{itemize}

\subsection{Goals}

The goals of this study are to evaluate trends in CDI and renal failure over the period of 2001-2014 using the NIS, and 
to determine risk factors of readmission for CDI patients over the period of 2010-2014 using the NRD.

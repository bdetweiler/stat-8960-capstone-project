\chapter{Contextual Data Quality}

Contextual data quality is assessed within the context of corresponding task at hand. \cite{tress} Good contextual dimensions require thorough planning before setting up and conducting the research, and they are difficult to be improved upon after the data has been collected.

\section{Adding Value}

Value-added information is not a static concept. \cite{philipp}. 
Expert coding, systematic indexing, timeliness, pricing, and customer service can all constituted as value-added. 
Philipp supports that everything that makes it easier, faster, and better value for the
money to find and deliver a relevant document adds value. 
Value-added information is considered more efficient and cheaper overall.
One can measure whether the information adds value by measuring the other 15 dimensions.
For example, when the timeliness of a data increase, it can be considered as adding value. 

In the stock market, people have the raw data from the news, financial reports, industry environment, etc. 
Thousands of data analysts need to analyze the data.
And the one who has the most accurate analysis in the shortest time as possible will net more profit.
This can be considered as value-added. 
    
To create value-added information, we need to be aware that it is not a static concept; 
the most important value changes along with the requirements of customers over time.
So, the organization need to be aware of the industry environment and provide the right services 
to the customers, so that the customers’ satisfaction will be good when their requirements evolve and develop. 


\section{Revelancy}

The accessibility of information has been increased with the development of the internet. However, information search precision
is quite low. \cite{peng} The traditional information retrieval method based on a Boolean model does not consider the 
user behavior and the keyword access frequency. 
Which means that, the method can only know which documents should be retrieved,
but cannot know which one is the more relevant if there are numbers of document retrieved.

Relevancy is a key in the information quality. The data captured should address the purposes for which it is to be used.
The information will be inadequate regardless of how well the information rates along the other 15 dimensions. 
This means that irrelevant information itself may have very good quality, but when it is used in the wrong place, 
it will be useless. For example, when a customer wants to know the price of a computer, 
the sales person gives him information of a television. No matter how good the television information is,
it is just a waste of time for both.

\section{Timeliness}

Data should be captured as quickly as possible after the event or activity and must be available for the intended use within a 
reasonable time period. Which means that data should be updated as soon as possible,
using the new data to replace the old data. Most of the articles support this in each industry, 
the information has a cycle time that depends on how quickly new information can be processed and communicated to its customers. 

Timeliness is very important in some industries, especially in the financial market.
Every time the investor gets the first-hand information, the investor will have more time to analyze the situation, 
and have the first mover advantage. A forecasting model based on an economic activity index that is subject to a short
publication lag is more efficient than other models. Economic forecasting models based on the ADS index derive their forecasting
power from the timeliness of the data releases used in its construction.
There will be more accurate forecasts if the ADS index is published with a lag of less than approximately 14 days.
Consequently, timeliness is very important in competitive industries. \cite{taylor} 

\section{Completeness}

Data requirements should be clearly specified based on the information needs of the organization 
and data collection processes matched to these requirements. There are four common causes of incompleteness. 
\begin{enumerate}
  \item Data are produced using subjective judgments, leading to omission
  \item Systemic errors in information production lead to lost data
  \item Access to data may conflict with requirements for security, privacy, and confidentiality
  \item Lack of sufficient computing resources limits access. \cite{wang} 
\end{enumerate}

It can be difficult simply finding a data set that contains complete entries. 
Some believe that the main reason for the cause of incompleteness is a requirement of different levels of information. 
For example, the CEO and the CIO of OPPD will both be interested in how the network is working inside the organization,
but the complete information for the CEO will be incomplete information for the CIO. 
Additionally, if the information exceeds a customer’s processing capability, it may be too complete. 
For example, if a new investor wants to buy a stock, the broker may give her some suggestions instead 
of a full package analysis of each stock.

\section{Amount of Data}

\lipsum 
